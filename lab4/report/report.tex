\documentclass[a4paper]{article}
\usepackage[utf8]{inputenc}
\usepackage[T1]{fontenc}
\usepackage[polish]{babel}
\usepackage{graphicx}
\usepackage{listings}
\usepackage{float}
\usepackage{geometry}
 \geometry{
 a4paper,
 total={170mm,257mm},
 left=20mm,
 top=20mm,
 }

\graphicspath{ {./images/} }

\author{Michał Stefanik}
\date{\today}
\title{Raport z zadania 3. - Generative Adversarial Networks}

\begin{document}

\maketitle
\section{Wstęp}
Do wykonania zadania używałem języka Python 3.10.13
z biblioteką PyTorch 2.1.0.
\section{Model dyskryminatora}

\begin{lstlisting}[language=python]
class Discriminator(nn.Module):
    def __init__(self):
        super(Discriminator, self).__init__()
        self.conv1 = nn.Conv2d(3, 32, kernel_size=4, stride=2, padding=2)
        self.norm1 = nn.BatchNorm2d(32)
        self.conv2 = nn.Conv2d(32, 64, kernel_size=4, stride=2, padding=2)
        self.norm2 = nn.BatchNorm2d(64)
        self.conv3 = nn.Conv2d(64, 128, kernel_size=4, stride=2, padding=2)
        self.norm3 = nn.BatchNorm2d(128)
        self.flatten = nn.Flatten()
        self.linear1 = nn.Linear(3200, 1)

    def forward(self, x):
        x = self.conv1(x)
        x = self.norm1(x)
        x = F.leaky_relu(x, 0.2)
        x = self.conv2(x)
        x = self.norm2(x)
        x = F.leaky_relu(x, 0.2)
        x = self.conv3(x)
        x = self.norm3(x)
        x = F.leaky_relu(x, 0.2)
        x = self.flatten(x)
        x = F.dropout(x, 0.5)
        x = self.linear1(x)
        x = F.sigmoid(x)
        return x
\end{lstlisting}

\section{Model generatora}

\begin{lstlisting}[language=python]
class Generator(nn.Module):
    def __init__(self):
        super(Generator, self).__init__()
        self.linear1 = nn.Linear(64, 64 * 4 * 4)
        self.conv1 = nn.ConvTranspose2d(64, 64, kernel_size=4, stride=2, padding=1)
        self.conv2 = nn.ConvTranspose2d(64, 128, kernel_size=4, stride=2, padding=1)
        self.conv3 = nn.ConvTranspose2d(128, 256, kernel_size=4, stride=2, padding=1)
        self.conv4 = nn.Conv2d(256, 3, kernel_size=5, stride=1, padding=2)

    def forward(self, x):
        x = self.linear1(x)
        x = x.view(-1, 64, 4, 4)
        x = self.conv1(x)
        x = F.leaky_relu(x, 0.2)
        x = self.conv2(x)
        x = F.leaky_relu(x, 0.2)
        x = self.conv3(x)
        x = F.leaky_relu(x, 0.2)
        x = self.conv4(x)
        x = torch.tanh(x)
        return x
\end{lstlisting}

Bez żadnego treningu, generator generuje obrazki przypominające szum. Przykład takigo obrazka wygenerowanego
z użyciem wektorów o wartościach z rozkładu normalnego standardowego na rysunku \ref{fig:noise}.

\begin{figure}[H]
    \centering
    \includegraphics[width=\textwidth]{noise.png}
    \caption{Przykładowy obrazek wygenerowany przez generator bez treningu}
    \label{fig:noise}
\end{figure}


\section{Zbiór danych}

Zbiór danych to dostarczone przez prowadzącego zdjęcia ciast marchewkowych.
Przykładowe zdjęcia zostały pokazane na rysunku \ref{fig:true_images_sample}.
Zdjęcia zostały przeskalowane do rozmiary 32 x 32, a ich wartości
przeskalowane do zakresu [-1, 1].

\begin{figure}[H]
    \centering
    \includegraphics[width=\textwidth]{true_images_sample.png}
    \caption{Przykładowe zdjęcia ciast marchewkowych}
    \label{fig:true_images_sample}
\end{figure}

\section{Przygotowanie się do działania z optimizerem}

Żeby mieć pewność, że optimizer będzie zmieniał tylko wagi dyskryminatora,
a nie generatora, zrobiłem mały ekspertyment. Na listingu \ref{lst:minimodel}
umieściłem kod prostego modelu z dwoma warstwami. Następnie stworzyłem
optimizer, który będzie optymalizował tylko pierwszą warstwę. Po wykonaniu
jednego i więcej kroków optymalizacji, wagi drugiej warstwy nie zmieniły się.
Wiemy więc, że optimizer zmienia jedynie wagi, które mu podamy jako parametr.

\begin{lstlisting}[language=python,label={lst:minimodel},caption={Eksperyment z optimizerem}]
class Minimodel(nn.Module):
    def __init__(self):
        super().__init__()
        self.lin = nn.Linear(2, 2)
        self.lin2 = nn.Linear(2, 2)

    def forward(self, x):
        x = self.lin(x)
        x = self.lin2(x)
        return x


minimodel = Minimodel()
input_val = torch.randn(10, 2)
loss_fn = lambda x: ((x - 42) ** 2).sum()
optimizer = torch.optim.SGD(minimodel.lin.parameters(), lr=0.01)

pred = minimodel(input_val)
loss = loss_fn(pred)
loss.backward()
optimizer.step()
\end{lstlisting}

Po wykonaniu kilku takich kroków wartość loss spadła znacząco, a wartości wyniku
modelu zbliżyły się do 42.

\section{Trening}
Do treningu zostały użyte następujące hiperparametry:
\begin{lstlisting}[language=python,caption={Parametry treningu}]
discriminator = Discriminator().to(device)
generator = Generator().to(device)

lrd = 0.00001
lrg = 0.000015
betas = (0.5, 0.999)

discriminator_optimizer = torch.optim.Adam(
    discriminator.parameters(), lr=lrd, betas=betas
)
generator_optimizer = torch.optim.Adam(
    generator.parameters(), lr=lrg, betas=betas
)
\end{lstlisting}

\subsection{Trening dyskryminatora}

Trening dyskryminatora składa się z dwóch części. W pierwszej części na wejściu
podajemy mu prawdziwe obrazki, które oznaczamy jako prawdziwe. W drugiej części
podajemy mu obrazki wygenerowane przez generator, które oznaczamy jako fałszywe.
Po zebraniu gradientu z obu części, wykonujemy jeden krok optymalizacji wag dyskryminatora.


\begin{lstlisting}[language=python]
# train discriminator on true images
discriminator_optimizer.zero_grad()
pred1 = discriminator(image_batch)
target = torch.ones((pred1.shape[0], 1), device=device)
loss1 = F.binary_cross_entropy(pred1, target)
loss1.backward()

# train discriminator on generated images
pred2 = discriminator.forward(
    generator.forward(torch.randn(16, 64, device=device))
)
target = torch.zeros((pred2.shape[0], 1), device=device)
loss2 = F.binary_cross_entropy(pred2, target)
loss2.backward()

discriminator_optimizer.step()
disc_losses.append((loss1 + loss2).item()/2)
\end{lstlisting}

\subsection{Trening generatora}

Trening generatora polegał na podaniu wygenerowanych przez niego obrazków
do dyskryminatora i oznaczeniu ich jako prawdziwe. Następnie zebraniu gradientu
i wykonaniu jednego kroku optymalizacji wag generatora.

\begin{lstlisting}[language=python]
# train generator
generator_optimizer.zero_grad()
batch = generator.forward(torch.randn(16, 64, device=device))
pred = discriminator(batch)
target = torch.ones((pred.shape[0], 1), device=device)
loss = F.binary_cross_entropy(pred, target)
loss.backward()
generator_optimizer.step()
gen_losses.append(loss.item())
\end{lstlisting}

\subsection{Monitoring modelu}

W celu monitorowania modelu co kilkadziesiat epok zapisywałem średnią
wartość funkcji loss dla dyskryminatora i generatora. Wyniki zostały
pokazane na rysunku \ref{fig:losses}. Oprócz kosztu zapisywałem też
obrazki wygenerowane przez generator dla kilku stałych wektorów wejściowych.
Zostały one zaprezentowane na rysunkach \ref{fig:100}, \ref{fig:500},
\ref{fig:1000} i \ref{fig:2000}. Dalszy trening nie przynosił znaczących
zmian w wizualnej jakości generowanych obrazków. Wartośći funkcji loss
zaczęły się znacząco różnić od siebie.


%% Tutaj wstawiony obrazek z lossami
\begin{figure}[H]
    \centering
    \includegraphics[width=\textwidth]{losses.png}
    \caption{Wartości funkcji loss dla dyskryminatora i generatora}
    \label{fig:losses}
\end{figure}

%% Tutaj obrazek po 100 epokach

\begin{figure}[H]
    \centering
    \includegraphics[width=\textwidth]{epoch_100.png}
    \caption{Obrazki wygenerowane przez generator po 100 epokach}
    \label{fig:100}
\end{figure}

%% Tutaj obrazek po 500 epokach

\begin{figure}[H]
    \centering
    \includegraphics[width=\textwidth]{epoch_500.png}
    \caption{Obrazki wygenerowane przez generator po 500 epokach}
    \label{fig:500}
\end{figure}

%% Tutaj obrazek po 1000 epokach
\begin{figure}[H]
    \centering
    \includegraphics[width=\textwidth]{epoch_1000.png}
    \caption{Obrazki wygenerowane przez generator po 1000 epokach}
    \label{fig:1000}
\end{figure}

%% Tutaj obrazek po 2000 epokach
\begin{figure}[H]
    \centering
    \includegraphics[width=\textwidth]{epoch_2000.png}
    \caption{Obrazki wygenerowane przez generator po 2000 epokach}
    \label{fig:2000}
\end{figure}

\section{Dopasowanie wejściowego wektora do obrazka}

Do tego celu użyłem modelu po 2000 epokach. W celu znalezienia wektora
wejściowego, obliczyłem gradient funkcji loss względem wektora wejściowego.
Wyniki dopasowania zostały pokazane na rysunku \ref{fig:fitting}.
Kod optymalizacji znajduje się na listingu \ref{lst:fitting}. Na rysunku
\ref{fig:fitting_evolution} została pokazana ewolucja obrazka od przypadkowego
do dopasowanego. Na rysunku \ref{fig:fitting_varying} zostały pokazane obrazki
wygenerowane z wektora wejściowego ze zmienianą jedną wartością.



\begin{lstlisting}[language=python,caption={Dopasowanie wektora wejściowego do obrazka},label={lst:fitting}]
input_vector = torch.randn(1, 64, device=device, requires_grad=True)
target_image = true_images_dataset[0]

image_optim = torch.optim.SGD([input_vector], lr=0.01, momentum=0.9)
loss_fn = lambda x: ((x - target_image) ** 2).sum()

for i in range(100):
    image_optim.zero_grad()
    result = generator.forward(input_vector)
    loss = loss_fn(result)
    loss.backward()
    image_optim.step()
    
\end{lstlisting}

\begin{figure}[H]
    \centering
    \includegraphics[width=\textwidth]{target_initial_result.png}
    \caption{Dopasowanie wektora wejściowego do obrazka}
    \label{fig:fitting}
\end{figure}

\begin{figure}[H]
    \centering
    \includegraphics[width=\textwidth]{intermediate_results.png}
    \caption{Ewolucja obrazka od przypadkowego do dopasowanego}
    \label{fig:fitting_evolution}
\end{figure}

\begin{figure}[H]
    \centering
    \includegraphics[width=\textwidth]{changing_parameters.png}
    \caption{Obrazki wygenerowane z wektora wejściowego ze zmienianą jedną wartością}
    \label{fig:fitting_varying}
\end{figure}

Dodatkowo procedura ta została powtórzona dla innego obrazka (grumpy cat),
niezwiązanego z ciastem marchewkowym. Wyniki zostały pokazane
na rysunku \ref{fig:fitting_other}.

\begin{figure}[H]
    \centering
    \includegraphics[width=\textwidth]{intermediate_cat_results.png}
    \caption{Dopasowanie wektora wejściowego do innego obrazka}
    \label{fig:fitting_other}
\end{figure}

\section{Interpolacja między obrazkami}

Ostatnim zadaniem było wygenerowanie interpolacji między dwoma obrazkami.
W tym celu do dwóch obrazków zostały dopasowane wektory wejściowe. Następnie
wygenerowano kilka obrazków leżących na linii w hiperpłaszczyźnie rozpiętej
na wektorach wejściowych. Wyniki zostały pokazane na rysunku \ref{fig:interpolation}.

\begin{figure}[H]
    \centering
    \includegraphics[width=\textwidth]{intermediate_results_2.png}
    \caption{Interpolacja między obrazkami}
    \label{fig:interpolation}
\end{figure}

\section{Wnioski i uwagi do zadania}

Zadanie na pewno satysfakcjonujące. Udało się wygenerować obrazki, które
jak się dobrze przypatrzy to przypominają ciasta marchewkowe. Niestety
nie udało się wygenerować obrazków, które byłyby jak dla mnie dobre jakościowo.
Dobór hiperparametrów był na pewno kluczowy do uzyskania dobrych wyników,
na co zeszło sporo czasu. Wartości funkcji loss dla dyskryminatora i generatora
zaczęły się znacząco różnić od siebie, co przyznam, że mnie zaskoczyło. Starałem
się to ograniczyć przez zmniejszenie współczynnika uczenia dyskryminatora,
pomogło wydłużyć równowagę, ale nie całkowicie. Najdłużej trwało uczenie -
w moim przypadku co najmniej 8 godzin, a fakt jakiegokolwiek uczenia było widać
dopiero po co najmniej godzinie. Wyniki nie są tak piękne jak te przykładowe,
ale może to być kwestia błędu w implementacji architektury sieci lub procesie uczenia.









\end{document}
